\documentclass[main.tex]{subfiles}

\begin{document}

\subsection{Differenciálegyenletek összefoglaló}



% ------------------------------------------------------------------------------
\begin{cbox}{yellow!40!gray}{Diffegyenletek osztályozása}
  Alakjuk lehet:
  \begin{itemize}
    \item $F(x;y;y';\dots;y^{(n)}) = 0$
          \tabto{5.35cm} – \tabto{6cm} implicit alak

    \item $y^{(n)} = f(x;y;y';\dots;y^{(n-1)})$
          \tabto{5.35cm} – \tabto{6cm} explicit alak
  \end{itemize}

  Változók alapján lehet:
  \begin{itemize}
    \item egyváltozós
          \tabto{2.6cm} – \tabto{3.2cm} közönséges

    \item többváltozós
          \tabto{2.6cm} – \tabto{3.2cm} parciális
  \end{itemize}

  A differenciálegyenlet rendje a legmagasabb derivált rendje
  \\[2mm]
  A diffegyenlet lineáris, ha $y$ deriváltjai csak az első
  hatványon szerepelnek, és szorzataik nem szerepelnek.
  Ekkor az alábbi alakban felírhatóak:
  \[
    a_n(x) y^{(n)}
    + a_{n-1}(x) y^{(n-1)}
    + \dots
    + a_1(x) y'
    + a_0(x) y
    = f(x)
  \]
\end{cbox}



% ------------------------------------------------------------------------------
\begin{cbox}{red!40!gray}{Szukcesszív approximáció}
  \[
    y_n := y(x_0) + \int_{x_0}^x f \left(
    t; y_{n-1}(t)
    \right) \differential t
  \]

  A módszert $n$ lépéses közelítésként használjuk.
  \\[2mm]
  Ha $f$ lipschitzes $\left(|f(x; y_1) + f(x; y_2)| \leq
    L |y_1 - y_2|\right)$, továbbá $|x-x_0|<a\leq\infty$
  és $|y-y_0|<b\leq\infty$ tartományon korlátos és
  folytonos, akkor $n \rightarrow \infty$ esetén $y_n$
  függvénysorozat az eredeti diffegyenlet megoldásához
  konvergál az $|x - x_0| < \min \{a; b/L\}$ intervallumon.
  \\[2mm]
  \emph{\underline{Példa:}}$\quad f(x; y) = y \quad y(0) = 1$
  \begin{alignat*}{3}
    y_1
     & = 1 + \int_0^x f(t; 1) \differential t
     &                                                & = 1 + \int_0^x 1 \, \differential t
     &                                                & = 1 + x
    \\
    y_2
     & = 1 + \int_0^x f(t; 1+t) \differential t
     &                                                & = 1 + \int_0^x 1+t \, \differential t
     &                                                & = 1 + x + x^2/2
    \\
    y_3
     & = 1 + \int_0^x f(t; 1+t+t^2/2) \differential t
     &                                                & = 1 + \int_0^x 1+t+t^2/2 \, \differential t
     &                                                & = 1 + x + x^2/2 + x^3/6
    \\
    \vdots
     &
    \\
    y_\infty
     & = \sum_{i=0}^\infty \frac{x^i}{i!} = e^x
  \end{alignat*}
\end{cbox}



% ------------------------------------------------------------------------------
\begin{cbox}{cyan!40!gray}{Szöveges feladatok I}
  \emph{\underline{Newton lehűlési törvénye:}}
  $\quad \dot{x} = \alpha(x - x_k)$
  \begin{gather*}
    \frac{\differential x}{x - x_k} = \alpha \, \differential t
    \\
    x(t) = x_k + c \cdot e^{\alpha t}
  \end{gather*}
  % \emph{\underline{Oldódási feladat:}}

  % TBD
\end{cbox}

\begin{cbox}{magenta!40!gray}{Elsőrendű, szétválasztható (szeparábilis) diffegyenletek}
  \[
    y'(x) = f(x)g(y)
  \]

  \emph{\underline{Megoldási menet:}}
  \[
    \derivative{y}{x} = f(x)g(y)
    \quad \rightarrow \quad
    \frac{\differential y}{g(y)} = f(x) \, \differential x
    \quad \rightarrow \quad
    \int \frac{\differential y}{g(y)} = \int f(x) \, \differential x
  \]

  \emph{\underline{Szétválaszthatóra visszavezethető esetek:}}
  \begin{alignat*}{3}
    y'
          & = f(y/x)
    \quad &                  & \rightarrow \quad
          &                  & u(x) = y/x
    \\
    y'
          & = f(ax + by + c)
    \quad &                  & \rightarrow \quad
          &                  & u(x) = ax + by + c
  \end{alignat*}

  \emph{\underline{Példa:}}$\quad (2x+1)y'-3y=0$
  \begin{gather*}
    y' = \frac{3y}{2x + 1} = \derivative{y}{x}
    \\
    \int \frac{\differential y}{3y} = \int \frac{\differential x}{2x + 1}
    \\
    \frac{1}{3} \ln |y| = \frac{1}{2} \ln |2x + 1| + K
    \\
    y = C(2x + 1)^{3/2}
  \end{gather*}
\end{cbox}



% ------------------------------------------------------------------------------
\begin{cbox}{magenta!40!cyan}{Elsőrendű, lineáris diffegyenletek}
  \[
    y' + p(x) y = q(x)
  \]
  Ha $q(x) \equiv 0 $, akkor az egyenlet homogén, egyébként inhomogén.
  \\[2mm]
  Az általános megoldás a homogén és inhomogén megoldás
  összegeként adódik:
  \begin{itemize}
    \item A homogén megoldást meghaphatjuk, ha megoldjuk az
          $y' + p(x)y = 0$ egyenletet a tanult módszerek alapján.

    \item Az inhomogén megoldást konstans variálással
          kaphatjuk meg. Ez azt jelenti, hogy a homogén megoldásban
          lévő konstanst $x$-től függőnek vesszük, majd megoldjuk
          az egyenletet.
  \end{itemize}
\end{cbox}



% ------------------------------------------------------------------------------
\begin{cbox}{red!60!green!80!gray}{Hiányos másodrendű egyenletek}
  Az $y'' = f(x; y; y')$ hiányos, ha $x$, $y$, vagy $y'$
  hiányzik az egyenletből.
  \begin{enumerate}
    \item $y'' = f(x)$
          \tabto{2.65cm} $\rightarrow$ \tabto{3.65cm}
          $y$ és $y'$ hiányos, kétszer integrálunk

    \item $y'' = f(x; y')$
          \tabto{2.65cm} $\rightarrow$ \tabto{3.65cm}
          $p(x) = y'(x)$ helyettesítés,

          \tabto{3.65cm}
          így $p$-re elsőrendű

    \item $y'' = f(y; y')$
          \tabto{2.65cm} $\rightarrow$ \tabto{3.65cm}
          $P(y) = y'$ helyettesítés,

          \tabto{3.65cm}
          ekkor $y'' = \partial_y P \cdot y' = P'_y + P$,

          \tabto{3.65cm}
          így $P$-re elsőrendű, szeparábilis
  \end{enumerate}
\end{cbox}



% ------------------------------------------------------------------------------
\begin{cbox}{cyan!20!black}{Egzakt diffegyenletek}
  \[
    P(x;y) \, \differential x + Q(x;y) \, \differential y = 0
  \]
  A fenti egyenlet egzakt, ha $\exists F(x;y)$, hogy
  $\partial_x F = P$ és $\partial_y F = Q$.

  Ekkor $F(x; y) = C$, $\partial_y P = \partial_x Q$
  teljsül. Ezt az $F$ függvényt keressük.
  \\[2mm]
  Előfordulhat azonban, hogy nem egzakt, vagyis
  $\partial_y P \neq \partial_x Q$, de $\exists M$
  egyváltozós multipli-kátor, mellyel egzakttá tehető.
  \begin{align*}
    \ln M(x)
     & = \int \frac{\partial_y P - \partial_x Q}{Q} \, \differential x
    \\
    \ln M(y)
     & = \int \frac{\partial_x Q - \partial_y P}{P} \, \differential x
  \end{align*}
\end{cbox}



% ------------------------------------------------------------------------------
\begin{cbox}{yellow!80!black}{Bernoulli-féle diffegyenletek}
  \[
    y' + a(x)y = b(x)y^\alpha
  \]
  \emph{\underline{Megoldás:}}
  $\quad u = q^{1-\alpha}$ helyettesítéssel
\end{cbox}

\begin{cbox}{cyan!80!black}{Ricatti-féle diffegyenletek}
  \[
    a(x) y' + b(x) y + c(x) y^2 = r(x)
  \]
  \emph{\underline{Megoldás:}}
  \quad Ismerni kell egy $y_p$ partikuláris megoldást.
  \begin{itemize}
    \item $y(x) = y_p(x) + z(x)$
          \tabto{4.4cm} – \tabto{5cm}
          Bernoulli-típusúra visszavezethető

    \item $y(x) = y_p(x) + 1/z(x)$
          \tabto{4.4cm} – \tabto{5cm}
          $z$-re lineáris
  \end{itemize}
\end{cbox}



% ------------------------------------------------------------------------------
\begin{cbox}{red!40!gray}{Lineáris differenciálegyenletek}
  \[
    y^{(n)}
    + a_{n-1}(x) y^{(n-1)}
    + \dots
    + a_1(x) y'
    + a_0(x) y
    = f(x)
  \]
  Ha $f(x) = 0$, akkor homogén, egyébként inhomogén.
  \\[2mm]
  A homogén általános megoldás alakja:
  \[
    y(x)
    = c_1 y_1(x)
    + c_2 y_2(x)
    + \dots
    + c_n y_n(x)
  \]
  Ez az $n$ darab függvény az integrálásra és
  deriválásra vektorteret alkot. Lineáris függetlenségüket
  A Wronsky-determimáns segítségével ellenőrizhetjük:
  \[
    \rmat{W} = \begin{bmatrix}
      y_1         & y_2         & \dots  & y_n         \\
      y_1'        & y_2'        & \dots  & y_n'        \\
      \vdots      & \vdots      & \ddots & \vdots      \\
      y_1^{(n-1)} & y_2^{(n-1)} & \dots  & y_n^{(n-1)} \\
    \end{bmatrix}
  \]
  Ha $\det \rmat{W} \neq 0$, akkor lineárisan függetlenek,
  $\left\{ y_1; y_2; \dots; y_n \right\}$ adják a diffegyenlet
  alaprendszerét. Az alaprendszer valamennyi függvénye, és
  lineáris kombinációjuk is megoldás lesz.
\end{cbox}



% ------------------------------------------------------------------------------
\begin{cbox}{green!30!cyan}{Állandó együtthatós, homogén, lineáris differenciálegyenletek}
  \[
    y^{(n)}
    + a_{n-1} y^{(n-1)}
    + \dots
    + a_1 y'
    + a_0 y
    = 0
  \]

  \emph{\underline{Megoldási módszer:}} \quad Próbafüggvény-módszer
  \\[2mm]
  Legyen $y = e^{\lambda x}$, $y' = \lambda e^{\lambda x}$,
  $\dots$, $y^{(n)} = \lambda^n e^{\lambda x}$. Ekkor az
  alábbi egyenletet kapjuk:
  \[
    \lambda^n
    + a_{n-1} \lambda^{n-1}
    + \dots
    + a_{1} \lambda^{1}
    + a_{0}
    = 0
  \]
  Ez $\lambda$-ra nézve $n$-edfokú polinom, melynek
  $n$ darab megoldása van, melyek között páros számú
  komplex gyök is lehet.
  \begin{itemize}
    \item $\lambda_1 \in \mathbb{R}$, egyszeres

          $y_1(x) = e^{\lambda_1 x}$

    \item $\lambda_1 = \lambda_2 = \dots = \lambda_s \in \mathbb{R}$,
          többszörös, akkor belső rezonancia áll fenn

          $y_1 = e^{\lambda_1 x}$,
          $y_2 = x e^{\lambda_1 x}$,
          $\dots$,
          $y_s = x^{s-1} e^{\lambda_1 x}$

    \item $\lambda_{1,2} = \alpha \pm i\beta$, egyszeres komplex gyökpár

          $y_1 = e^{\alpha x} \cos (\beta x)$

          $y_2 = e^{\alpha x} \sin (\beta x)$

    \item $\lambda_{2s} \in \mathbb{C}$, $s$-szeres multiplicitású

          $y_1 = e^{\alpha x} \cos (\beta x)$
          \tabto{4cm}
          $y_3 = x y_1$
          \tabto{7cm}
          $y_5 = x^2 y_1$
          \tabto{10cm}
          $\dots$
          \tabto{11.5cm}
          $y_{2s-1} = x^{s-1} y_1$

          $y_2 = e^{\alpha x} \sin (\beta x)$
          \tabto{4cm}
          $y_4 = x y_4$
          \tabto{7cm}
          $y_6 = x^2 y_2$
          \tabto{10cm}
          $\dots$
          \tabto{11.5cm}
          $y_{2s} = x^{s-1} y_2$
  \end{itemize}
\end{cbox}



% ------------------------------------------------------------------------------
\begin{cbox}{green!60!blue}{Állandó együtthatós, inhomogén, lineáris differenciálegyenletek}
  \[
    y^{(n)}
    + a_{n-1} y^{(n-1)}
    + \dots
    + a_1 y'
    + a_0 y
    = f(x)
  \]

  \emph{\underline{Megoldási módszer:}} \quad
  Megoldást a gerjesztésnek megfelelő
  formában kell keresni.
  \begin{itemize}
    \item $f(x) = e^x$
          \tabto{4.6cm} – \tabto{5.2cm}
          $y_\mathrm{ih} = A e^x$

    \item $f(x) = p(x)$, polinom
          \tabto{4.6cm} – \tabto{5.2cm}
          $y_\mathrm{ih} = A_0 + A_1 x + A_2 x^2 + \dots$

    \item $f(x)$ trigonometrikus
          \tabto{4.6cm} – \tabto{5.2cm}
          $y_\mathrm{ih} = A \cos (\beta x) + B \sin (\beta x)$
  \end{itemize}
  \[
    y = y_\mathrm{h} + y_\mathrm{ih}
  \]
\end{cbox}



% ------------------------------------------------------------------------------
\begin{cbox}{magenta!30!red}{Euler-féle diffegyenletek}
  \[
    a_2 \, y'' + \frac{a_1}{x} \, y' + \frac{a_0}{x^2} \, y = 0
  \]
  \emph{\underline{Megoldási módszer:}}
  \quad $x = e^z$ helyettesítéssel
  \quad ($\dot{y} = \partial_z y$)
  \[
    y'  = \frac{\dot{y}}{e^{z}}
    \quad\quad\quad
    y'' = \frac{\ddot{y} - \dot{y}}{e^{2z}}
  \]
\end{cbox}



% ------------------------------------------------------------------------------
\begin{cbox}{green!30!red}{Másodrendű, változó együtthatós diffegyenletek}
  \[
    a_2(x) y'' + a_1(x) y' + a_0(x) y = 0
  \]
  Ha ismert az egyik megoldás, akkor konstansvariációval
  meg tudjuk adni a másikat.
  \[
    y_2 = c(x) y_1
  \]
\end{cbox}



% ------------------------------------------------------------------------------
\begin{cbox}{cyan!40!gray}{Szöveges feladatok II}
  \emph{\underline{Általános keverési feladat:}}

  \begin{minipage}[c]{.5\textwidth}
    \begin{figure}[H]
      \centering
      \begin{tikzpicture}
        \filldraw[snake=coil, segment aspect=0, very thick, cyan!40!gray, fill=cyan!20] (-3, 2)
        -- ++(6, 0)
        [snake=none, black, fill=cyan!20]-- ++(0,-2)
        -- ++(-2.9, 0)
        -- ++(0,-0.5)
        -- ++(-0.2, 0)
        -- ++(0,0.5)
        -- ++(-2.9, 0)
        -- ++(0, 2)
        ;
        \draw[snake=coil, segment aspect=0, ultra thick, cyan!40!gray] (-3,2) -- ++(6,0);

        \draw[very thick] (-3, 3) -- (-3, 0) -- (-0.1, 0) -- (-0.1, -0.5);
        \draw[very thick] (3, 3) -- (3, 0) -- (0.1, 0) -- (0.1, -0.5);

        \draw[cyan!40!gray!5!white, ultra thick] (0.2, -0.5) -- ++(-.4, 0);

        \draw[red!40!gray, ultra thick, -to] (0, 3.75) -- (0, 2.5)
        node [midway, black] {$p_{be} \;\;\; \dot{V}_{be}$}; % \dot{V}_{be}
        \draw[red!40!gray, ultra thick, -to] (0, -0.25) -- (0, -1.5)
        node [midway, black] {$p_{ki} \;\;\; \dot{V}_{ki}$}; % \dot{V}_{ki}

        \node[] at (0, 1) {$V_0$};
      \end{tikzpicture}
    \end{figure}
  \end{minipage}\hfill
  \begin{minipage}[c]{.5\textwidth}
    Ha a bejövő és a kiáramló anyagmennyiség megegyezik
    ($\dot{V}_{be} = \dot{V}_{ki} = \dot{V}_{e} \in \mathbb{R}$), akkor:
    \[
      x_{be} = \dot{V}_{e} \cdot p_{be}
      \quad \quad \quad
      x_{ki}(t) = \frac{\dot{V}_e}{V_0} \; x(t)
    \]
    \begin{gather*}
      \dot{x}(t) = x_{be} - x_{ki}(t)
      \\[2mm]
      \frac{\differential x}{x_{be} - x_{ki}(t)} = \differential t
      % \\[2mm]
      % x(t) = x_{be} + c e^{-t}
    \end{gather*}

    Ha a bejövő és a kiáramló anyagmennyiség nem egyezik meg,
    azaz $\dot{V}_{be} \neq \dot{V}_{ki}$, ekkor:
    \[
      x_{ki}(t) = \frac{\dot{V}_{ki}}{V_0 + (\dot{V}_{be} - \dot{V}_{ki})t} \; x(t)
    \]
  \end{minipage}
\end{cbox}

\end{document}
