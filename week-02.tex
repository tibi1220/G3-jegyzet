\documentclass[main.tex]{subfiles}

\begin{document}

\megj{1}{.5}{$\varphi^{*}$ mátrix reprezentációja}

Reprezentálja $\varphi$-t $\rmat{A}$,
$\varphi^{*}$-ot pedig $\rmat{A}^{*}$:
\begin{align*}
  \scalar{\varphi(\rvec{v})}{\rvec{w}}
  \; & = \;
  \scalar{\rvec{v}}{\varphi^{*}(\rvec{w})}
  \\[
  2mm
  ]
  \rmat{A}\rvec{v} \cdot \rvec{w}
     & =
  \rvec{v} \cdot \rmat{A}^{*} \rvec{w}
  \\[
  2mm
  ]
  \begin{bmatrix}
    a_{11} & a_{12} \\
    a_{21} & a_{22}
  \end{bmatrix}
  \begin{bmatrix}
    v_1 \\
    v_2
  \end{bmatrix}
  \cdot
  \begin{bmatrix}
    w_1 \\
    w_2
  \end{bmatrix}
     & =
  \begin{bmatrix}
    v_1 \\
    v_2
  \end{bmatrix}
  \cdot
  \begin{bmatrix}
    a_{11}^{*} & a_{12}^{*} \\
    a_{21}^{*} & a_{22}^{*}
  \end{bmatrix}
  \begin{bmatrix}
    w_1 \\
    w_2
  \end{bmatrix}
  \\[
  2mm
  ]
  \begin{bmatrix}
    a_{11} v_1 + a_{12} v_2 \\
    a_{21} v_1 + a_{22} v_2
  \end{bmatrix}
  \cdot
  \begin{bmatrix}
    w_1 \\
    w_2
  \end{bmatrix}
     & =
  \begin{bmatrix}
    v_1 \\
    v_2
  \end{bmatrix}
  \cdot
  \begin{bmatrix}
    a_{11}^{*} w_1 + a_{12}^{*} w_2 \\
    a_{21}^{*} w_1 + a_{22}^{*} w_2
  \end{bmatrix}
\end{align*}
\begin{alignat*}{4}
  \scalar{\rmat{A}\rvec{v}}{\rvec{w}}
   & = \circled{$a_{11}$}{green!40!gray} v_1 w_1
   & + \circled{$a_{12}$}{red!40!gray} v_2 w_1
   & + \circled{$a_{21}$}{cyan!40!gray} v_1 w_2
   & + \circled{$a_{22}$}{yellow!40!gray} v_2 w_2
  \\
  \scalar{\rmat{A}^{*}\rvec{w}}{\rvec{v}}
   & = \circled{$a_{11}^{*}$}{green!40!gray} w_1 v_1
   & + \circled{$a_{12}^{*}$}{cyan!40!gray} w_2 v_2
   & + \circled{$a_{21}^{*}$}{red!40!gray} w_1 v_2
   & + \circled{$a_{22}^{*}$}{yellow!40!gray} w_2 v_2
\end{alignat*}

Megállapthatjuk, hogy $\rmat{A}^{*} = \rmat{A}^\transpose$.


\allit{1}{.33} Szimmetrikus leképezés adjungáltja önmaga.


\allit{1}{.33}{Leképezés felbontása}

$! \; \varphi \in \End(V)$, ekkor $!\exists$ olyan
$\mathcal{A}$ és $\mathcal{S}$ antiszimmetrikus és szimmetrikus
leképezés, ahol $\varphi = \mathcal{A} + \mathcal{S}$, melyek
az endomorfizmusok vektorterét 2 diszjunkt halmazra bontják:
\begin{equation*}
  \mathcal{A} := \frac{\varphi - \varphi^{*}}{2}
  \hspace{5mm} \text{és} \hspace{5mm}
  \mathcal{S} := \frac{\varphi + \varphi^{*}}{2}
\end{equation*}


\biz{0}{.33}{unicitás}

Tegyük fel hogy $\varphi$ előáll
$\mathcal{A}_1 + \mathcal{S}_1$ és
$\mathcal{A}_2 + \mathcal{S}_2$
összegeként is. Vonjuk ki egymásból
a két egyenletet:
\begin{align*}
  \mathcal{O}
  = \varphi - \varphi
   & = \overbrace{ \left(
    \mathcal{A}_1 - \mathcal{A}_2
    \right)}^{\displaystyle\overline{\mathcal{A}}}
  + \overbrace{ \left(
    \mathcal{S}_1 - \mathcal{S}_2
    \right)}^{\displaystyle\mathcal{\overline{S}}}
  \\[
  2mm
  ]
  \mathcal{O}
   & = \mathcal{\overline{A}} + \mathcal{\overline{S}}
  \hspace{25mm}
  /\ (\;\;\;)^{*}
  \hspace{5mm}
  (\text{adjugált leképezés})
  \\[
  2mm
  ]
  \mathcal{O}^{*}
  = \mathcal{O}
   & = \mathcal{\overline{A}^{*}} + \mathcal{\overline{S}^{*}}
  = -\mathcal{\overline{A}} + \mathcal{\overline{S}}
\end{align*}

Az előző két egyenletből következik, hogy
$\mathcal{O = A = S}$. Feltevésünk hamisnak bizonyult.


\ism{1}{.33}{Reguláris mátrix felbontása}

Egy $\rmat{M} \in \mathcal{M}_{n \cross n}$
mátrix felbontható szimmetrikus és
ferdeszimmetrikus (antiszimmetrikus) részekre:
\begin{equation*}
  \rmat{S} = \frac{\rmat{M} + \rmat{M}^\transpose}{2}
  \hspace{20mm}
  \rmat{A} = \frac{\rmat{M} - \rmat{M}^\transpose}{2}
\end{equation*}

Ha mátrixunk $3 \cross 3$-as:
\begin{equation*}
  \rmat{M}
  = \rmat{S} + \rmat{A}
  = \begin{bmatrix}
    a & d & e \\
    d & b & f \\
    e & f & c
  \end{bmatrix}
  + \begin{bmatrix}
    0  & x  & y \\
    -x & 0  & z \\
    -y & -z & 0
  \end{bmatrix}
\end{equation*}

\begin{itemize}
  \item Általános esetben
        egy ferdeszimmetrikus
        mátrix dimenziója:
        $\dfrac{n \cdot (n - 1)}{2}$

  \item Szimmetrikus mátrix dimenziója:
        $\dfrac{n \cdot (n + 1)}{2}$
\end{itemize}


\megj{0}{.33}

Az antiszimmetrikus leképezések és a
$V$-beli vektorok között tudunk egy-egyértelmű
hozzárendelést találni:
\begin{equation*}
  \rmat{A} \in \mathcal{A}
  \hspace{2.5mm} \leftrightarrow \hspace{2.5mm}
  \rvec{v} \in V
\end{equation*}

Keressünk egy olyan $\rvec{v}$ vektort,
melyre teljesül az alábbi egyenlet:
\begin{align*}
  \rmat{A}\rvec{w}
   & = \rvec{v} \cross \rvec{w}
  \\[
  2mm
  ]
  \begin{bmatrix}
    0       & a_{12}  & a_{13} \\
    -a_{12} & 0       & a_{23} \\
    -a_{13} & -a_{23} & 0
  \end{bmatrix}
  \begin{bmatrix}
    w_1 \\
    w_2 \\
    w_3
  \end{bmatrix}
   & =
  \begin{bmatrix}
    v_1 \\
    v_2 \\
    v_3
  \end{bmatrix}
  \cross
  \begin{bmatrix}
    w_1 \\
    w_2 \\
    w_3
  \end{bmatrix}
  \\[
  2mm
  ]
  \begin{bmatrix}
    \rected{$+a_{12}$}{red}  w_2 \rected{$+a_{13}$}{blue}  w_3 \\
    \rected{$-a_{12}$}{red}  w_1 \rected{$+a_{23}$}{green} w_3 \\
    \rected{$-a_{13}$}{blue} w_1 \rected{$-a_{23}$}{green} w_2
  \end{bmatrix}
   & =
  \begin{bmatrix}
    \rected{$v_2$}{blue}  w_3 \rected{$-v_3$}{red}   w_2 \\
    \rected{$v_3$}{red}   w_1 \rected{$-v_1$}{green} w_3 \\
    \rected{$v_1$}{green} w_2 \rected{$-v_2$}{blue}  w_1
  \end{bmatrix}
  \hspace{20mm}
  \\[
  2mm
  ]
  \rvec{v}
   & =
  \begin{bmatrix}
    -a_{23} \\
    a_{13}  \\
    -a_{12} \\
  \end{bmatrix}
\end{align*}


\defi{0}{.33}{vektorinvariáns}

Egy antiszimmetrikus lineáris transzformáció
mindig leírható egy rögzített vektorral
való keresztszorzással. Ez a vektor a leképezés
vektorinvariánsa.


\megj{1}{.33}

Csak ortonormált, Descartes-féle bázisban
számítható az előbbi módszerrel egy leképezés
vektorinvariánsa.


\allit{1}{.33}{Lineáris transzformáció nyoma}

Egy lineáris transzformáció főátlójában lévő
elemek összege minden koordinátarendszerben
ugyanannyi, tehát a koordináta-transzformáció
nem befolyásolja. Ezt nevezzük a lineáris
leképezés nyomának. (trace / spur)

%-------------------------------------------------------------------------------
%-------------------------------- Subsection 1.4 -------------------------------
%-------------------------------------------------------------------------------
\subsection{Differenciáloperátorok}

Legyen $f: V \rightarrow V$ függvény ($\dim V = n$),
melynek vegyük a deriváltját:
\begin{equation*}
  f \left( x_1; \; x_2; \; \dots ; \; x_n \right)
  =
  \begin{bmatrix}
    f_1 \left( x_1; \; x_2; \; \dots ; \; x_n \right) \\
    f_2 \left( x_1; \; x_2; \; \dots ; \; x_n \right) \\
    \vdots                                            \\
    f_n \left( x_1; \; x_2; \; \dots ; \; x_n \right) \\
  \end{bmatrix}
\end{equation*}
\begin{equation*}
  \Diff f
  =
  \begin{bmatrix}
    \partial_1 f_1 & \partial_2 f_1 & \dots  & \partial_n f_1 \\
    \partial_1 f_2 & \partial_2 f_2 & \dots  & \partial_n f_2 \\
    \vdots         & \vdots         & \ddots & \vdots         \\
    \partial_1 f_n & \partial_2 f_n & \dots  & \partial_n f_n
  \end{bmatrix}
  =
  \begin{bmatrix}
    \grad^\transpose f_1 \\
    \grad^\transpose f_2 \\
    \vdots               \\
    \grad^\transpose f_n \\
  \end{bmatrix}
  \in
  \mathcal{M}_{n \cross n}
\end{equation*}

Definiáljuk az alábbi fogalmakat:
\begin{itemize}
  \item rotáció
        \tabto{2.4cm} – \tabto{3cm}
        $\rot f := \Diff f - \Diff f ^{*}$

  \item divergencia
        \tabto{2.4cm} – \tabto{3cm}
        $\div f := \tr \left( \Diff f \right)$
\end{itemize}

$V = \mathbb{R}^3$ esetén:
\begin{equation*}
  \rot f
  =
  \begin{bmatrix}
    0                               &
    \partial_2 f_1 - \partial_1 f_2 &
    \partial_3 f_1 - \partial_1 f_3
    \\
    \partial_1 f_2 - \partial_2 f_1 &
    0                               &
    \partial_3 f_2 - \partial_2 f_3
    \\
    \partial_1 f_3 - \partial_3 f_1 &
    \partial_2 f_3 - \partial_3 f_2 &
    0
  \end{bmatrix}
\end{equation*}

A mátrix vektorinvariánsa:
\begin{equation*}
  \rvec{v}
  =
  \begin{bmatrix}
    \partial_2 f_3 - \partial_3 f_2 \\
    \partial_3 f_1 - \partial_1 f_3 \\
    \partial_1 f_2 - \partial_2 f_1
  \end{bmatrix}
  =
  \underbrace{
    \begin{bmatrix}
      \partial_1 \\
      \partial_2 \\
      \partial_3
    \end{bmatrix}
  }_{\displaystyle\nabla}
  \cross
  \begin{bmatrix}
    f_1 \\
    f_2 \\
    f_3
  \end{bmatrix}
\end{equation*}

$\nabla$ - Nabla operátor (formális differenciáloperátor)
– nem vektor, de vektorként viselkedik.

\pelda{1}{0}{Rotáció és divergencia számítása}
\pagebreak


\jel{0}{0}
\begin{itemize}
  \item $\grad f = \nabla f$
  \item $\div f = \; \scalar{\nabla}{f}$
  \item $\rot f = \nabla \cross f$
\end{itemize}


\pelda{0}{0}{Differenciáloperátorok kombinálása}
\begin{itemize}
  \item $\grad \grad$,
        $\grad \rot$,
        $\div \div$,
        $\rot \div$
        – nem értelmes.

  \item $\div \rot \rvec{u} \equiv 0$

  \item $\rot \grad \varphi \equiv \rvec{0}$

  \item $\div \grad \varphi
          = \laplacian \, \varphi
          = \Delta \, \varphi$
        \hspace{1em} $\rightarrow$ \hspace{1em}
        Laplace operátor
\end{itemize}

\pagebreak


\end{document}