\documentclass[main.tex]{subfiles}

\begin{document}

\subsection{Integrálátalakító tételek}

\tetel{0}{.33}{Stokes – tétel}

! $\rsurf{S} :
  [\, a; b \,] \cross [\, a; b \,]
  \rightarrow \mathbb{R}^3$ irányított, parametrizált,
elemi felület. Irányítsuk a peremet jobbkézszabály szerint.
Továbbá ! $\rvec{v} : \mathbb{R}^3 \rightarrow \mathbb{R}^3$
legalább egyszer folytonosan differenciálható vektormező,
ekkor
\begin{equation*}
  \int \limits_\rsurf{S}
  \scalar{\rot \rvec{v}}{\differential \vsurf{S}}
  =
  \oint \limits_{\rcurve{\gamma}}
  \scalar{\rvec{v}}{\differential \rvec{r}},
\end{equation*}
ahol $\rcurve{\gamma} = \partial \rsurf{S}$.



\megj{1}{0}
\begin{itemize}
  \item A tétel elemi felületre van kimondva.


  \item A tétel a következőképpen is kimondható:

        Határolja $\rsurf{S}$ felületet $\rcurve{\gamma}$
        zárt görbe. Legyen egyszeresen differenciálható
        (pontra húzható), itányítható. Ekkor a vektormező
        a határvonalán folytonosan differenciálható, a
        fenti összefüggés érvényes.


  \item Ha a teljes felület nem írható le $z = f(x;y)$
        függvénnyel, akkor felbontjuk olyan részekre,
        melyeken már érvényes a tétel. Ott alkalmazzuk,
        majd a részeredményeket összegezzük.

  \item Egy vektormező bármely zárt görbe menti integrálja
        akkor tűnik el, ha $\rvec{v} = \grad \varphi$,
        hiszen $\rot \rvec{v} = \rot \grad \varphi \equiv 0$.
\end{itemize}



\defi{1}{.33}{Elemi tértartomány}

$\rvol{V} : [\, a; b \,]^3 \rightarrow \mathbb{R}^3$
elemi tértartomány, ha legalább egyszer folytonosan
differenciálható leképezés. Ekkor
$\det \left( \Diff \rsurf{V} \right) \neq 0$.



\defi{1}{.33}{Térfogat}

Készítsünk egy olyan beleírt ($c_i$) és körülírt ($C_i$)
kockarendszert, melyekre igaz, hogy $c_i \cap c_j$
csak lap, él, vagy csúcs lehet. Ekkor fennáll, hogy:
\begin{equation*}
  \underset{i}{\cup} \; c_i
  \subset \rvol{V} \subset
  \underset{j}{\cup} \; C_i
\end{equation*}

Finomítsuk minden határon túl ezeket a kockarendszereket.
Ha ezek közös határértékhez tartanak, akkor:
\begin{equation*}
  \vol V = \iiint \limits_{[\, a; b \,]^3} \left|
  \det \left( \Diff \rsurf{V}(u; v; t) \right)
  \right|
  \, \differential u
  \, \differential v
  \, \differential t
\end{equation*}



\megj{1}{.33}

Pozitív az irányítás, ha $\det \Diff \rvol{V} > 0$.



\defi{1}{.33}{Skalármező térfogaton vett skalárértékű integrálja}

! $\varphi : \mathbb{R}^3 \rightarrow \mathbb{R}$ folytonos
skalármező, $\rvol{V}$ irányított, paraméterezett, elemi
tértartomány. Ekkor a $\varphi$ $\rvol{V}$-n vett integrálja:
\begin{equation*}
  \int \limits_{\rvol{V}} \varphi \, \differential \rvol{V}
  =
  \iiint \limits_{[\, a; b \,]^3} \varphi \left(
  \rvol{V}(u; v; t)
  \right) \det \left(
  \Diff \rvol{V}(u; v; t)
  \right)
  \, \differential u
  \, \differential v
  \, \differential t
\end{equation*}



\tetel{1}{.33}{Gauss – Osztogradszkij tétel}

! $\rvol{V} : [\, a; b \,]^3 \rightarrow \mathbb{R}^3$
elemi, irányított, paraméterezett tértartomány,
$\rvec{w} : \mathbb{R}^3 \rightarrow \mathbb{R}^3$
$\rvol{V}$-n legalább egyszer folytonosan differenciálható
vektormező. Jelölje $\partial \rvol{V} = \rsurf{S}$
a $\rvol{V}$ peremét indukált irányítással. Ekkor:
\begin{equation*}
  \int \limits_{\rvol{V}} \div \rvec{w} \, \differential \rvol{V}
  =
  \oint \limits_{\partial \rvol{V}}
  \scalar{\rvec{w}}{\differential \vsurf{S}}
\end{equation*}



\megj{1}{0}
\begin{itemize}
  \item Ha a vektormező forrásmentes, akkor bármely
        felületre vett felületi integrálja zérus.

  \item A térfogatot továbbonthatjuk kisebb egységekre,
        majd a részeredményeket összegezzük.
\end{itemize}



\tetel{1}{.33}{Green – tétel asszimmetrikus alakja}

! $\varphi, \psi : \mathbb{R}^3 \rightarrow \mathbb{R}$
kétszeresen folytonos skalármezők, $\rvol{V} \subset \mathbb{R}^3$
paraméterezett, irányított tértartomány,
$\partial \rvol{V} = \rsurf{S}$
perem indukált irányítással, ekkor:
\begin{equation*}
  \int \limits_{\rvol{V}}
  \psi \, \Delta \varphi \; + \scalar{\grad \psi}{\grad \varphi}
  \, \differential \rvol{V}
  =
  \int \limits_{\partial \rvol{v}}
  \scalar{\psi \grad \varphi}{\differential \vsurf{S}}
\end{equation*}



\biz{1}{.33}

! $\rvec{w} = \psi \grad \varphi$, akkor:
\begin{equation*}
  \div \rvec{w}
  =
  \div \left(
  \psi \grad \varphi
  \right)
  =
  \psi \, \Delta \varphi \; + \scalar{\grad \psi}{\grad \varphi}
\end{equation*}



\tetel{1}{.33}{Green – tétel szimmetrikus alakja}
! $\varphi, \psi : \mathbb{R}^3 \rightarrow \mathbb{R}$
kétszeresen folytonos skalármezők, $\rvol{V} \subset \mathbb{R}^3$
paraméterezett, irányított tértartomány,
$\partial \rvol{V} = \rsurf{S}$
perem indukált irányítással, ekkor:
\begin{equation*}
  \int \limits_{\rvol{V}}
  \psi \, \Delta \varphi - \varphi \, \Delta \psi
  =
  \int \limits_{\partial \rvol{V}}
  \scalar{
    \psi \grad \varphi - \varphi \grad \psi
  }{\differential \vsurf{S}}
\end{equation*}

\end{document}