\documentclass[main.tex]{subfiles}

\begin{document}

\defi{1}{.33}{Kompakt halmaz}

A $H$ halmazt kompakt halmaznak mondjuk, ha $\forall$
nyílt lefedésből kiválasztható véges sok nyílt halmaz,
melynek uniója lefedi a $H$-t.



\megj{1}{.33}{Nyílt és zárt halmazok}

\begin{itemize}
  \item Nyílt halmaz – Bármely halmazbeli pontnak $\exists$ olyan
        $\varepsilon$ sugarú környezete, hogy az ebben lévő pontok
        mindegyike a halmaz részei.

  \item Zárt halmaz – Olyan halmaz, melynek komplementere nyílt
\end{itemize}



\allit{0}{.33}

$D = I_1 \cross I_2 \cross \dots \cross I_n$, ahol
$I_i$, $i \in \left\{1;2;\dots;n\right\}$ zárt halmaz
kompakt.




\tetel{1}{.33}

! $y' = f(x; y)$, $y(x_0) = y_0$ differenciálegyenlet-rendszer, és
$\left( x_0; y_0 \right) \in D$, ahol $D$ zárt, kompakt tégla (halmaz)
$\mathbb{R}^2$-ben. Ekkor ha a $\varphi$ függvény megoldása a
Cauchy-feladatnak, akkor $\varphi$ biztosan elhagyja a téglát.



\defi{1}{.33}{Iránymező és vonalelemek}

% \begin{minipage}[c]{.4\textwidth}
%   \begin{figure}[H]
%     \centering
%     \begin{tikzpicture}[scale = .5]
%       % \draw[very thin,color=gray] (-5.1,-5.1) grid (5.1,5.1);
%       \draw[->] (-5.2,0) -- (5.2,0) node[right] {$x$};
%       \draw[->] (0,-5.2) -- (0,5.2) node[above] {$y$};

%       \draw[thick, cyan!50!black, domain=-4.8:-0.2, smooth, dashed] plot ({\x},{1/\x});
%       \draw[thick, cyan!50!black, domain=0.2:4.8  , smooth, dashed] plot ({\x},{1/\x});

%       \draw[thick, cyan!50!black, domain=-4.8:-0.4, smooth, dashed] plot ({\x},{2/\x});
%       \draw[thick, cyan!50!black, domain=0.4:4.8  , smooth, dashed] plot ({\x},{2/\x});

%       \draw[thick, cyan!50!black, domain=-4.8:-0.6, smooth, dashed] plot ({\x},{3/\x});
%       \draw[thick, cyan!50!black, domain=0.6:4.8  , smooth, dashed] plot ({\x},{3/\x});

%       \draw[thick, cyan!50!black, domain=-4.8:-0.8, smooth, dashed] plot ({\x},{4/\x});
%       \draw[thick, cyan!50!black, domain=0.8:4.8  , smooth, dashed] plot ({\x},{4/\x});
%     \end{tikzpicture}
%   \end{figure}
% \end{minipage}\hfill
% \begin{minipage}[c]{.6\textwidth}
% \end{minipage}

A vonalelemek a megoldásgörbe adott pontbeli érintői.

Ha az iránymező állandó, vagyis a megoldásagörbe
állandó meredekségű, akkor izonklinákról beszélünk.



\pelda{1}{20}



%-------------------------------------------------------------------------------
%-------------------------------- Subsection 2.2 -------------------------------
%-------------------------------------------------------------------------------
\subsection{Szeparábilisra visszavezethető differenciálegyenletek}

\megj{1}{0}{Szeparábilis diffegyenlet megoldása}
\begin{gather*}
  y' = f(x)g(y)
  \\
  \int \frac{1}{g(y)} \differential y = \int f(x) \, \differential x
\end{gather*}



\defi{0}{.33}{Homogén függvény}

Az $f : \mathbb{R}^2 \rightarrow \mathbb{R}$ függvényt
az $x$ és $y$ változók $k$-adrendű homogén függvényének
mondjuk, ha $f \left( tx; ty \right) = t^k f(x; y)$
$\forall x;y \in D_f$ $\forall t \in \mathbb{R}$.

\vspace{1.5em}


\begin{minipage}[c]{.31\textwidth}
  \begin{cbox}{gray}{1. eset}
    \[
      y' = f \left( ax + by + c \right)
    \]
    \vfill
  \end{cbox}
\end{minipage}\hfill
\begin{minipage}[c]{.31\textwidth}
  \begin{cbox}{gray}{2. eset}
    \[
      y' = f \left( \frac{y}{x} \right)
    \]
    Változóiban homogén
    \vfill
  \end{cbox}
\end{minipage}\hfill
\begin{minipage}[c]{.31\textwidth}
  \begin{cbox}{gray}{3. eset}
    \[
      y' = \left( \frac{
        a_1 x + b_1 y + c_1
      }{
        a_2 x + b_2 y + c_2
      } \right)
    \]
  \end{cbox}
\end{minipage}\hfill

\begin{enumerate}
  \item eset: Helyettesítéssel.
        \begin{gather*}
          y' = f \left( ax + by + c \right) \\[3mm]
          u(x) = a x + b y + c \\
          u'(x) = a + b y' \\
          u' = a + b f(u) \\
          \hspace{4em}
          \frac{\differential u}{a + b f(u)}
          = \differential x
          \quad \rightarrow \quad {\textstyle\int \dots}
        \end{gather*}

  \item eset: Ha $k = 0$, akkor $t = 1/x$ helyettesítéssel.
        \begin{gather*}
          y' = f \left( \frac{y}{x}\right)
          \quad \rightarrow \quad
          u(x) = \frac{y(x)}{x}
          \\[2mm]
          \derivative{u}{x}
          = \frac{y' \cdot x - y}{x^2}
          = \frac{y' - y/x}{x}
          = \frac{f(u) - u}{x}
          \\[2mm]
          \frac{\differential u}{f(u) - u}
          = \frac{\differential x}{x}
          \quad \rightarrow \quad {\textstyle\int \dots}
        \end{gather*}


  \item eset: További 2 esetre bomlik tovább.
        \begin{enumerate}
          \item Ha $c_1 = c_2 = 0$, vagyis $c_1^2 + c_2^2 = 0$.
                \begin{equation*}
                  y' = f \left(
                  \frac{ a_1 x + b_1 y }{a_2 x + b_2 y}
                  \right) = f \left(
                  \frac{a_1 + b_1 (y / x)}{a_2 + b_2 (y / x)}
                  \right) = f \left(
                  \frac{y}{x}
                  \right)
                \end{equation*}

          \item Ha $c_1 = c_2 \neq 0$, akkor helyettesítéssel.
                \begin{equation*}
                  \left.
                  \begin{array}{l}
                    x := \xi + p \\
                    y := \eta + q
                  \end{array}
                  \right\}
                  \rightarrow
                  \quad
                  \derivative{y}{x} = \derivative{\eta}{\xi}
                  \quad \left( p; q \in \mathbb{R} \right)
                \end{equation*}
                \begin{equation*}
                  y' = f \left(
                  \frac{
                    a_1 \xi + b_1 \eta + a_1 p + b_1 q + c_1
                  }{
                    a_2 \xi + b_2 \eta + a_2 p + b_2 1 + c_2
                  }
                  \right)
                \end{equation*}

                \vspace{1em}
                \begin{multicols}{2}
                  \begin{enumerate}
                    \item Ha $\det\left[\begin{smallmatrix}
                                a_1 & b_1 \\ a_2 & b_2
                              \end{smallmatrix} \right] \neq 0$

                          \vspace{.66em}
                          Határozzuk meg $p$ és $q$ értékét úgy,
                          hogy $a_1 p + b_1 q + c_1 = 0$ és
                          $a_2 p + b_2 q + c_2 = 0$ feltételek
                          teljesüljenek. Ekkor:
                          \[
                            y' =  f \left(
                            \frac{a_1 \xi + b_1 \eta}{a_2 \xi + b_2 \eta}
                            \right)
                          \]
                          Így már változóiban homogén,
                          szeparábilisre visszavezethető.

                    \item Ha $\det\left[\begin{smallmatrix}
                                a_1 & b_1 \\ a_2 & b_2
                              \end{smallmatrix} \right] = 0$

                          \vspace{.66em}
                          Ebből következik, hogy $a_1 b_2 = a_2 b_1$,
                          ekkor $a_2 / a_1 = b_2 / b_1 = k$.

                          Legyen $a_2 = k a_1$ és $b_2 = k b_1$,
                          ekkor:
                          \[
                            y' = f \left(
                            \frac{
                              a_1 x + b_1 y + c_1
                            }{
                              k a_1 x + k b_2 y + c_2
                            }
                            \right)
                          \]
                          Így már változóiban homogén,
                          szeparábilisre visszavezethető.
                  \end{enumerate}
                \end{multicols}
        \end{enumerate}
\end{enumerate}







\end{document}