\documentclass[main.tex]{subfiles}

\begin{document}

\defi{1}{.33}{Elemi felület}

Egy $S: [\; a \; ; \; b \;] \cross [\; a \; ; \; b \;]
  \rightarrow \mathbb{R}^3$ elemi felület, ha $S$ legalább
egyszer differenciálható és injektív.



\megj{1}{0}
\begin{itemize}
  \item $\partial \left(
          \; [\; a \; ; \; b \;] \cross [\; a \; ; \; b \;] \;
          \right)$ a paramétertartomány pereme.

  \item A felület irányítható, ha megadható rajta egy
        differenciálható egységvektormező.
\end{itemize}
\begin{minipage}[c]{.49\textwidth}
  \begin{figure}[H]
    \centering
    \begin{tikzpicture}
      \begin{axis}[
          view={-45}{45}
        ]
        \addplot3[
          opacity = 0.8,
          surf,
          colorbar,
          colormap/PuBu,
          shader=interp,
          domain=-1:1,
          domain y=-1.3:1.3,
          samples=10
        ]{exp(-x^2-y^2)};

        \draw[-latex, ultra thick, green!40!magenta] (0,0,1) -- (0,0,1.5);
        \draw[-latex, ultra thick, green!80!magenta] (0,0,1) -- (0,0,0.5);

        \draw[-latex, ultra thick, green!40!magenta] (-.25,.25,.88) -- (-.5,0.25,1.4);
        \draw[-latex, ultra thick, green!80!magenta] (-.25,.25,.88) -- (0,0.25,.36);

        % \draw[-latex, ultra thick, green!40!magenta] (-.25,-.25,.88) -- (-.5,-.35,1.4);
        \draw[-latex, ultra thick, green!40!magenta] (.25,-.25,.88) -- (.5,-.25,1.2);
        \draw[-latex, ultra thick, green!80!magenta] (.25,-.25,.88) -- (0 ,-.25,.56);

        \node [small dot,pin=210:{Belső pont}] at (-0.5, -0.5, 0.6 )  {};
        \node [small dot,pin=90:{Perempont}]   at (0   , -1.3, 0.18)  {};
      \end{axis}
    \end{tikzpicture}
    \caption{Irányítás szemléltetése}
  \end{figure}
\end{minipage}
\begin{minipage}[c]{.49\textwidth}
  \begin{figure}[H]
    \centering
    \begin{tikzpicture}[
        declare function={f(\x,\y)=exp(-\x^2-\y^2);}
      ]
      \begin{axis}[
          view={135}{45},
        ]
        \addplot3[
          opacity = 1,
          surf,
          colorbar,
          colormap/PuBu,
          % shader=interp,
          domain=-1:1,
          domain y=-1.3:1.3,
          samples=14
        ]{f(x,y)};

        \draw[-latex, ultra thick, blue!40!magenta]   (0,0,1) -- (-1,0,.9)
        node[
          midway,
          above left=.2cm and -0.1cm,
          black,fill=yellow!50!white,
          rectangle,
          rounded corners=3pt
        ] {$\partialderivative{\rvec{r}}{u}$};

        \draw[-latex, ultra thick, yellow!40!magenta] (0,0,1) -- (0,1,1.1)
        node[
          midway,
          below=.2cm,
          black,fill=yellow!50!white,
          rectangle,
          rounded corners=3pt
        ] {$\partialderivative{\rvec{r}}{v}$};
        s

        \draw[dotted, ultra thick, blue!40!magenta] (0,1,1.1) -- (-1,1,1);
        \draw[dotted, ultra thick, yellow!40!magenta] (-1,0,.9) -- (-1,1,1);

        \draw[fill=white!90!yellow,fill opacity=0.6]
        (0,0,1) -- (-1,0,.9) -- (-1,1,1) -- (0,1,1.1) -- cycle
        node [above right=-0.2cm and 1.2cm, opacity=1] {$A$};
      \end{axis}
    \end{tikzpicture}
    \caption{Elemi felület}
  \end{figure}
\end{minipage}




\defi{1}{.33}{Skalármező skalárértékű felületmenti integrálja}

Legyen $f: \mathbb{R}^3 \rightarrow \mathbb{R}$ skalármező,
$\vsurf{\varphi}: U \subset \mathbb{R}^2 \rightarrow \mathbb{R}^3$
Ekkor finomítsuk minden határon túl a
\begin{equation*}
  \sum_i f \left( \xi_i ;\; \eta_i ;\; \zeta_i \right) \Delta \varphi_i
\end{equation*}
összeget:
\begin{equation*}
  \int f ( \rvec{\varphi}(\rvec{r})) \, \differential\varphi
\end{equation*}
Számítása:
\begin{gather*}
  \iint f(\rvec{r}(u ;\; v)) \norma{
    \partialderivative{\rvec{r}}{u}
    \cross
    \partialderivative{\rvec{r}}{v}
  } \, \differential u \, \differential v
  \quad \rightarrow \quad
  \text{ha a felület paraméterezve van}
  \\
  \iint f(x ;\; y ;\; \varPhi(x ;\; y)) \sqrt{
    1
    + \left( \partial_x \varPhi \right)^2
    + \left( \partial_y \varPhi \right)^2
  } \, \differential x \, \differential y
  \quad \rightarrow \quad
  \text{ha } z=\varPhi(x ;\; y) \text{ alakban van}
\end{gather*}


\pelda{1}{20}{Súlypont számítása}


\defi{1}{.33}{Skalármező vektorértékű felületmenti integrálja}

Felület implicite megadása esetén
$\left( z = \varPhi(x ;\; y) \right)$:
\begin{equation*}
  \int f(\rvec{r}) \, \differential \vsurf{S}
  =
  \iint f(x ;\; y ;\; \varPhi(x ;\; y))
  \begin{bmatrix}
    \pm \partial_x \varPhi \\
    \pm \partial_y \varPhi \\
    \mp 1
  \end{bmatrix}
  \, \differential x \, \differential y
\end{equation*}



\defi{1}{.33}{Vektormező skalárértékű felületmenti integrálja}

Legyen $\rvec{w}: \mathbb{R}^3 \rightarrow \mathbb{R}^3$ vektormező,
$\rvec{r}: U \subset \mathbb{R}^2 \rightarrow \mathbb{R}^3$ felület:

\begin{minipage}[c]{.49\textwidth}
  \begin{figure}[H]
    \centering
    \begin{tikzpicture}[
        declare function={f(\x,\y)=exp(-\x^2-\y^2);}
      ]
      \begin{axis}[
          view={100}{45},
          axis lines=none,
        ]
        \addplot3[
          opacity = 1,
          surf,
          colorbar,
          colormap/PuBu,
          % shader=interp,
          domain=-1:1,
          domain y=-1.3:1.3,
          samples=12
        ]{f(x,y)};

        \draw[-latex, ultra thick, green!40!magenta]   (0,0,1) -- (0,0,1.6)
        node[
          midway,
          left=.2cm,
          black,fill=yellow!50!white,
          rectangle,
          rounded corners=3pt
        ] {$\rvec{n}_i$};

        \draw[-latex, ultra thick, yellow!40!magenta] (0,0,1) -- (-.33,1.33,1.2)
        node[
          midway,
          below=.2cm,
          black,fill=yellow!50!white,
          rectangle,
          rounded corners=3pt
        ] {$\rvec{w}_i$};
      \end{axis}
    \end{tikzpicture}
    \caption{Normálvektor szemléltetése}
  \end{figure}
\end{minipage}
\begin{minipage}[c]{.5\textwidth}
  Finomítsuk minden határon túl a
  \begin{equation*}
    \sum_i \scalar{
      \rvec{w}(\xi_i ;\; \eta_i ;\; \zeta_i)
    }{
      \rvec{n}_i(\xi_i ;\; \eta_i ;\; \zeta_i)
    }
  \end{equation*}
  összeget:
  \begin{gather*}
    \int \scalar{\rvec{w}}{\differential\vsurf{S}} \; =
    \\
    = \iint \scalar{
      \rvec{w}(\rvec{r}(u ;\; v))
    }{
      \partialderivative{\rvec{r}}{u}
      \cross
      \partialderivative{\rvec{r}}{v}
    }
    \, \differential u \, \differential v
  \end{gather*}
\end{minipage}



\defi{1}{.33}{Vektormező vektorértékű felületmenti integrálja}
\begin{equation*}
  \int \rvec{w} \cross \differential \vsurf{S}
  = \iint \rvec{w}(\rvec{r}(u ;\; v)) \cross
  \left(
  \partialderivative{\rvec{r}}{u}
  \cross
  \partialderivative{\rvec{r}}{v}
  \right) \, \differential u \, \differential v
\end{equation*}

\end{document}