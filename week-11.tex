\documentclass[main.tex]{subfiles}

\begin{document}

%-------------------------------------------------------------------------------
%-------------------------------- Subsection 2.3 -------------------------------
%-------------------------------------------------------------------------------
\subsection{Elsőrendű lineáris differenciálegyenletek}

Az $a(x) y' + b(x) y + c(x)$ alakú differenciálegyenletet,
ahol $a;b;c : I \subset \mathbb{R} \rightarrow \mathbb{R}$,
$a(x) \not\equiv 0$
egyváltozós, valós, folytonos függvények, elsőrenű
lineáris diffegyenletnek nevezzük.

\vspace{.33em}
Gyakran a következő alakban írjuk del őket:
\begin{equation*}
  y' + p(x)y = q(x)
\end{equation*}

Megoldásuk:
\begin{itemize}
  \item Homogén rész
        \tabto{3cm} – \tabto{3.6cm}
        Ha $q(x) \equiv 0$,
        akkor az egyenlet az alábbi alakra redukálódik:
        \begin{gather*}
          y' + p(x)y = 0
          \\
          \frac{1}{y} \, \differential y = -p(x) \, \differential x
          \\
          \ln y = -\int p(x) \, \differential x + \ln C
          \\
          \quad\quad\quad
          y_h = C e^{-\int p(x) \, \differential x}
          \quad C \in \mathbb{R}
        \end{gather*}

  \item Ingomogén rész
        \tabto{3.1cm} – \tabto{3.6cm}
        Ha $q(x) \not\equiv 0$,
        akkor ezt zavarófüggvénynek nevezzük.
        % \tabto{3.6cm}

        A megoldás menete a következő:
        \begin{enumerate}
          \item Megoldjuk a homogén egyenletet
          \item $y_h(x) = c Y(x)$, ahol $Y(x)
                  = e^{-\int p(x) \, \differential x} $
          \item Konstans variáció (Lagrange-féle módszer) alkalmazása:
                \begin{align*}
                  y_p(x)  & = c(x)Y(x)
                  \\
                  y'_p(x) & = c'(x)Y(x) + c(x)Y'(x)
                \end{align*}
                Visszahelyettesítünk az ereddeti egyenletbe:
                \begin{gather*}
                  y'_p(x) + p(x)y_p(x) = q(x)
                  \\
                  c'(x)Y(x) + c(x)Y'(x) + p(x)c(x)Y(x) = q(x)
                  \\
                  c(x) \underbrace{\left(Y'(x) + p(x)Y(x) \right)}_{
                  = 0 \text{, homogén egyenlet}
                  } + c'(x)Y(x) = q(x)
                \end{gather*}
                $c(x)$ meghatározása:
                \begin{gather*}
                  c'(x) = \frac{q(x)}{Y(x)}
                  \\
                  c(x) = \int \frac{q(x)}{Y(x)} \,\differential x
                \end{gather*}
                Az általános megoldás a homogén és partikuláris megoldás
                összegeként áll elő:
                \begin{equation*}
                  y_\text{ált}(x)
                  = y_h(x) + y_p(x)
                  = c Y(x) + c(x) Y(x)
                  = e^{-\int p(x) \, \differential x} \left(
                  \int \frac{q(x)}{Y(x)} \, \differential x + k
                  \right)
                \end{equation*}
        \end{enumerate}
\end{itemize}



\megj{0}{.33}

Partikuláris megoldást próbafüggvény-módszerrel
is kereshetünk, erről később lesz szó.



%-------------------------------------------------------------------------------
%-------------------------------- Subsection 2.4 -------------------------------
%-------------------------------------------------------------------------------
\subsection{Bernoulli-féle differenciálegyenletek}

Az olyan egyenleteket, melyek
\begin{equation*}
  y'(x) + p(x)y(x) = q(x)y^n(x)
\end{equation*}

alakban írhatóak fel, és
$p; q : I \subset \mathbb{R} \rightarrow \mathbb{R}$
folytonos függvények, valamint $n \not\in \left\{0; 1\right\}$,
Bernoulli-féle diffegyenleteknek nevezzük. Ezek az egyenletek
nem lineárisak, de az alábbi helyettesítéssel lineárissá tehetőek:
\begin{gather*}
  z(x) = y^{1-n}(x)
  \\
  z'(x) = (1 - n) y^{-n} y'
  \\
  y^{-n} y' = \frac{z'}{1-n}
\end{gather*}
Az eredeti egyenlet mindkét oldalát osszuk el $y^n$-nel,
majd végezzük el a helyettesítést:
\begin{gather*}
  y^{-n} y' + p(x) y^{1 - n} = q(x)
  \\
  \frac{z'(x)}{1 - n} + p(x) z(x) = q(x)
\end{gather*}
Ez már $z$-re lineáris, megoldható.



%-------------------------------------------------------------------------------
%-------------------------------- Subsection 2.5 -------------------------------
%-------------------------------------------------------------------------------
\subsection{Ricatti-féle differenciálegyenletek}

A Ricatti-típusú diffegyenletek elsőrendű, másodfokú,
nemlineáris egyenletek, amelyek az alábbi alakban
írhatóak fel:
\begin{equation*}
  y'(x) + p(x)y(x) = q(x)y^2(x) + h(x)
\end{equation*}
Szükségünk van egy partikuláris megoldásra.
Ennek segítségével helyettesítsünk az alábbiak szerint:
\begin{itemize}
  \item $y(x) = \dfrac{1}{z(x)} + y_p(x)$
        \tabto{3.9cm} – \tabto{4.4cm}
        Így már $z$-re lineáris.

  \item $y(x) = z(x) + y_p(x)$
        \tabto{3.9cm} – \tabto{4.4cm}
        Bernoulli-típusú lesz.
\end{itemize}



%-------------------------------------------------------------------------------
%-------------------------------- Subsection 2.5 -------------------------------
%-------------------------------------------------------------------------------
\subsection{Egzakt differenciálegyenletek}

A $P(x;y) \,\differential x + Q(x;y) \,\differential y = 0$,
diffegyenletet – ahol $P;Q : \mathbb{R}^2 \rightarrow
  \mathbb{R}$ folytonosan differenciálható függvények
– egzaktnak nevezzük, ha teljesül az alábbi feltétel:
\begin{equation*}
  \partialderivative{P(x;y)}{y} = \partialderivative{Q(x;y)}{x}
\end{equation*}
Ekkor $\exists F : \mathbb{R}^2 \rightarrow \mathbb{R}$,
melyre igaz, hogy:
\begin{gather*}
  \partialderivative{F(x;y)}{x} = P(x;y)
  \quad \text{és} \quad
  \partialderivative{F(x;y)}{y} = Q(x;y)
  \\
  \partialderivative{F(x;y)}{x}   \, \differential y
  + \partialderivative{F(x;y)}{y} \, \differential x
  = 0
  \\
  F(x;y) = C
\end{gather*}
Előfordulhat, hogy a $P(x;y) \,\differential x
  + Q(x;y) \,\differential y = 0$ diffegyenlet
nem egzakt, de integráló tényező segítségével
egzakttá tehető. Tegyük fel, hogy $\exists M :
  \mathbb{R}^2 \rightarrow \mathbb{R}$, hogy
$M(x;y) \, P(x;y) \, \differential x + M(x;y)
  \, Q(x;y) \, \differential y = 0$ diffegyenlet
már egzakt, azaz:
\begin{gather*}
  \partialderivative{MP}{y} = \partialderivative{MQ}{x}
  \\[.33em]
  \partialderivative{M}{y} P
  + M \partialderivative{P}{y}
  = \partialderivative{M}{x} Q
  + M \partialderivative{Q}{x}
  \\[.33em]
  Q \partialderivative{M}{x}
  - P \partialderivative{M}{y}
  + M \left(
  \partialderivative{Q}{x}
  - \partialderivative{P}{y}
  \right) = 0
\end{gather*}
Tegyük fel, hogy $M$ egyváltozós: $!M(x)$, ekkor
a diffegyenlet az alábbi alakra redukálódik:
\begin{equation*}
  Q \partialderivative{M}{x}
  + M \left(
  \partialderivative{Q}{x}
  - \partialderivative{P}{y}
  \right) = 0
\end{equation*}
Ez már $M$-re szeparábilis:
\begin{gather*}
  \frac{1}{M} \, \differential M
  = \frac{1}{Q} \left(
  \partialderivative{P}{y}
  - \partialderivative{Q}{x}
  \right) \differential x
  \\[.33em]
  \ln M
  = \int \frac{P_y' - Q_x'}{Q} \, \differential x + \ln C
  \\[.33em]
  M(x) = C e^{\int \frac{P_y' - Q_x'}{Q} \, \differential x}
\end{gather*}
Itt $C$ tetszőlegesen megválasztható, ezért
célszerű $1$-nek választani.

\end{document}