\documentclass[main.tex]{subfiles}

\begin{document}

%-------------------------------------------------------------------------------
%-------------------------------- Subsection 2.1 -------------------------------
%-------------------------------------------------------------------------------
\subsection{Elsőrendű differenciálegyenletek}

\jel{0}{0}
\begin{itemize}
  \item $F \left(x ; y ; y'\right)$
        \tabto{2.5cm} – \tabto{3.1cm}
        Implicit-alak

  \item $y' = f \left(x ; y\right)$
        \tabto{2.5cm} – \tabto{3.1cm}
        Explicit-alak

  \item $Q \left(x ; y\right) \differential y
          + R \left(x ; y\right) \differential x$
\end{itemize}



\tetel{0}{0}{Cauchy-féle egzisztencia és unicitás tétele}

Tegyügy fel, hogy az $f: \mathbb{R}^2 \rightarrow \mathbb{R}$
$\left( x_0 ; y_0 \right)$ esetén létezik olyan
$\left( x_0 ; y_0 \right) \in D$ tartomány, hogy
\begin{itemize}
  \item $f(x; y)$ folytonos $D$-n mindkét változójában,
  \item $f'_y$ létezik, és korlátos $D$-n.
\end{itemize}
Ekkor $\exists!$ megoldása az $y' = f(x; y)$, $y(x_0) = y_0$
kezdeti feltétellel ellátott differenciálegyenletek, azaz
$\exists!$ $\varphi :
  \left( x_0 - \varepsilon; \; x_0 + \varepsilon \right)
  \rightarrow \mathbb{R}$ folytonosan differenciálható függvény,
hogy $\varphi' (x) = f \left(x ; \varphi(x)\right)$
$\forall x \in \left( x_0 - \varepsilon; \; x_0 + \varepsilon \right)$
esetén teljesül, és $\varphi(x_0) = y_0$.



\megj{1}{0}
\begin{itemize}
  \item A $D$ tartomány minden egyes pontján egyetlen megoldásgörbe megy át.
  \item A tételben foglaltfeltételek nem szükségesek, de elégségesek.
\end{itemize}



\pelda{0}{20}



\defi{1}{.33}{Lipschitz – feltétel}

Az $f : \mathbb{R}^2 \rightarrow \mathbb{R}$ függvényre azt monjuk,
hogy  a $D$ tartományon az $y$ változóra nézve kielégíti a
Lipschitz-feltételt, ha $\exists$ olyan $M \in \mathbb{R}^+$, hogy
$\forall$ $\left( x; y_0 \right)$ és $\left( x; y_1 \right)$ esetén
\begin{equation*}
  \left| f(x; y_0) - f(x; y_1) \right| \leq M \left| y_0 - y_1 \right|.
\end{equation*}



\tetel{1}{.33}{Picard – Lindelöf tétel}

! $y' = f(x, y)$ adott és $D = I_1 \cross I_2$,
ahol $I_1$ és $I_2$ nyílt intervallumok,
$\left( x_0 \; ; \; y_0 \right) \in D$.
Tegyük fel hogy:
\begin{itemize}
  \item $f$ mindkét változójában folytonos $D$-n,
  \item $f$ kielégíti a Lipschitz-feltételt az $y$
        változójára nézve.
\end{itemize}
Ekkor az $y' = f(x; y)$, $y(x_0) = y_0$ kezdeti feltétellel
ellátott differenciálegyenletmek $\exists!$ megoldása, azaz
$\exists \varepsilon > 0$, hogy $\varphi :
  \left( x_0 - \varepsilon; \; x_0 + \varepsilon \right)
  \rightarrow \mathbb{R}$-re teljesül, hogy $\varphi' (x)
  = f \left(x ; \varphi(x)\right)$
$\forall x \in \left( x_0 - \varepsilon; \; x_0 + \varepsilon \right)$
esetén $\varphi(x_0) = y_0$.



\tetel{1}{.33}{Peano – tétel}

Ha az $f$ függvényről csak folytonosságot tesszük fel,
akkor azt mondhatjuk, hogy van legalább egy integrálgörbéje,
amely átmegy az $\left( x_0 ; y_0 \right)$ ponton.



\megj{1}{0}
\begin{itemize}
  \item Látható tehát, hogy az $f$ folytonossága minden
        Cauchy-feladat megoldásához elégséges, de az egyértelmű
        megoldás létezéséhez nem.

  \item Az egyértelmű megoldás létezéséhez az $f$
        lipschitzessége a folytonossággal elégséges,
        de nem szükséges feltétel.
\end{itemize}



\tetel{1}{.33}{Szukcesszív approximáció}

Ha az $y = f(x; y)$ differenciálegyenletben lévő
$f$ függvényre teljesül, hogy
$\left| x - x_0 \right| < a \leq \infty$ és
$\left| y - y_0 \right| < b \leq \infty$ tartományon
korlátos és folytonos, továbbá eleget tesz a
Lipschitz-feltételnek, akkor a
\begin{equation*}
  y_n
  := \underbrace{y(x_0)}_{y_0}
  + \int_{x_0}^x f\left(
  t; y_{n-1}(t)
  \right) \differential t
\end{equation*}
függvénysorozat $n \rightarrow \infty$ esetén az
$y' = f(x; y)$, $y(x_0) = y_0$ differenciálegyenlet
megoldásához konvergál az  $\left| x - x_0 \right|
  < \min \left\{ a; b/M \right\}$  intervallumon.




\end{document}